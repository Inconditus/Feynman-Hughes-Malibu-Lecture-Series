\documentclass[a4paper,10pt]{article}
\usepackage[utf8x]{inputenc}

%opening

\begin{document}
\begin{titlepage} 

\title{{\LARGE \textbf{Feynman Hughes Malibu\\ Lecture Series}} \\
FMHL Volume 5\\
\textbf{Mathematical Techniques of \\ Engineering \& Physics}\\
Oct. 1970-June 1971
Weekly 2 hour lectures\\
Notes taken and transcribed by \\ John T. Neer}
\author{}
\date{}

\maketitle

\end{titlepage}

\renewcommand\contentsname{Lecture}
\tableofcontents
\pagebreak

\section{Introduction to the Lectures On Mathematical Techniques of Engineering and Physics}

Today begins a new year of lectures and this time we will discuss the subject of mathematical methods of engineering and physics.
 We will be discussing such topics as complex
 notation, vectors, tensors, various special functions like Bessel functions which come about in non-linear problems, Fourier transforms
and series. 

Our problem is not to formulate the physics of a given problem but rather to apply useful mathematical techniques in solving
the resulting equations.  As such our efforts will not be directed towards generalizing our results to a broad class of problems
but rather to pay attentions to certain aspects of the problem in question. The first and foremost task before us is to formulate
an approximate equation which contains the physics and can be approximately solved. It is important to appreciate that we are 
not striving for the exact answer but only for an approximately right answer.  This must be kept in mind as we progress; always 
remember the accuracy of the desired answer.

\begin{center}
 Solving an equation

We all know how to solve the simple quadratic equation
$$ x^2+2x+7=0$$
And we should know how to solve the cubic equation
$$x^3+3x^2+x-1=0$$
but how do we solve for x given the equation
$$e^x=cosx?$$
\end{center}

These are simple examples which could be categorized as a set of equations which are:
\begin{enumerate}
 \item Linear, and therefore trivial.
\item Quadratic and therefore trivial. 
\item And all the rest which are non trivial.
\end{enumerate}

It is the third category which interests the most since in the real world all the problems are found in it.

Lets take an example of a cubic equation and see how we might solve it.  Given the cubic which we write as,

\begin{center}$\frac{1}{1+x^2}=2x$
\end{center}
we will solve it by the trial and error method.  If this method gives us the right answer-Great! There is nothing
wrong with writing down some numbers in the process. This doesn't represent a cultural lag; it is sick to think that
it is. 

In order to help us solve the equation we will invent some ways to increase our efficiency in guessing the right answer.
Let's then form a table of values for x, the left hand side of the equation and the right we will also compute the difference
between the two sides. The following steps are taken
\begin{center}
\begin{tabular}{p{5cm} c c c r}
 \centering Step & X & L.H.S. & R.H.S. & Diff\\
 & & & & {\small LHS-RHS}\\

1. Try x=0 & 0 & 1.000 & .000 & 1.000\\
2. That didn't work ; try x=1 & 1 & 0.50 & 2.00 & -1.500\\
3. That didn't work either but we are on either side of the answer. We might guess the answer
lies .4 of the way between 0 and 1 so lets try x=.5 & .5 & .8 & 1.00 & -2.00\\
4. We're getting better try x=.4 & .4 & .862 & +.800 & +.62\\
5. Since the difference in 3 and 4 is on either side of the right answer interpolate between them, i.e. $\Delta=\frac{62}{262}
=.24$ so try x=.424 & .424 & .847 & .848 &.001
\end{tabular}
\end{center}

And now we have the answer to an accuracy of .1\%. So this method is pretty accurate and is better than a machine because
it can't guess what to do next.

This method of interpolation works when the difference between the two sides of the equation come out to be + and -. 
When that happens you can interpolate and go again.  One word of caution never use graph paper; it is always easier to use the
numbers.

There are two more methods which are more appropriate of machine than by hand; they are the method of iteration and Newton's method.
The idea is to write the equation as
\begin{center}
 $$x_{out}=\frac{1}{2}\frac{1}{1+x^2_{in}}$$
\end{center}

where you now try an $x_{in}$ value, .e.g. $x_{in}=0$ and find $x_{out}$; then plug that value back in, etc.

\begin{tabular}{c c c}
step & $x_{in}$ & $x_{out}$\\
1 & 0 & .50\\
2 & .5 & .4\\
3 & .4 & .431\\
4 & .431 & .426
\end{tabular}

If you want a lot of accuracy in the answer this method has an error which decreases more slowly than the previous method. 
Also you should be careful that the equation is written in the right form otherwise the answer won't converge. To show this solve
for an $x_{in}$ 
\begin{center}
 $$x_{in}=\sqrt{\frac{1}{2x_{out}}-1}$$
\end{center}

Make a table and start evaluating
\begin{center}
\begin{tabular}{c c}
$x_{in}$ & $x_{out}$\\
.4 & .5\\
.5 & 0.0\\
0 & $\infty$

\end{tabular}
\end{center}

To see why the right answer is diverging lets substitute \begin{center}$x_{in}=x_{true}+\epsilon_{in}$\end{center}
where $X_T=\sqrt{\frac{1}{2x_T-1}}$

Then $$\sqrt{\frac{1}{2(x_T+\epsilon_{in})}-1}=\sqrt{(\frac{1}{2x_T-1})-\frac{\epsilon_{in}}{4x_o^2}}=x_o\sqrt{1+\frac{\epsilon_{in}}{4x_o^4}}$$.

Expanding we have $x_o-\frac{\epsilon_{in}}{8x_o^3}$

So that $$\epsilon_{out}=-\frac{1}{8x_o^3}\epsilon_{in}$$

$\epsilon_{out}$ is greater than twice the error in so the answer is diverging. 

\begin{center}
Newton's Method
\end{center}
The Newton's method requires writing the function in the form
$$f(x)=0$$ i.e. 
$$f(x)=\frac{1}{1+x^2} - 2x = 0$$

If you are close to the answer with a guess, say $x_{1}$ then evaluate $f(x_{1})$ and also $f'(x_{1})$. In this case
$$f'(x) = \frac{2x}{(1+x^2)^2} - 2$$

Then the next try would be

$$x_{2}=x_{1}-\frac{f(x_{1})}{f'(x_{1})}$$

Try $x_{1}=0$ then $f(x_{1})=0-\frac{1}{2}=.5$ and $$f'(x_{1})=\frac{.2}{2.64}$$
and $x_{2}=.5-\frac{.2}{2.64}=.424$

This technique doesn't require much intelligence so it is great for a computer. 
The answer will always converge except for a few rare examples. 
However this method involves evaluating both $f(x)$ and $f'(x)$. 
$f'(x)$ may be hard to compute and difficult to evaluate.

Problem: Solve $e^{-x}=cos(x)$ to 1%

A word in complex roots. If you made a mistake and didn't do the problem right or you didn't expect
the physics right the roots may be complex. To find complex roots you have to solve two equations in two unknowns. 
This can be done by the same procedure outlined before. Given $f(x,y)=0$ and $g(x,y)=0$ find $y$ as a function of 
$x$ if possible then substitute back into one of the equations.

\end{document}